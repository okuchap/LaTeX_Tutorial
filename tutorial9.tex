\documentclass[11pt]{article}

\begin{document}

The function $f(x)=(x-3)^2+\frac{1}{2}$ has domain $\mathrm{D}_f:(-\infty,+\infty)$ and range $\mathrm{R}_f:\left[\frac{1}{2},\infty\right)$ \\
%\mathrm{•}でイタリック回避

$\lim \limits_{x \to a}f(x)$ \\

$\lim \limits_{x \to a^-}f(x)$ \\

$\lim_{x \to a}$ \\ %極限の表示をlimの下にするか横にするか

$\lim \limits_{x \to a}\frac{f(x)-f(a)}{x-a}=f'(x)$

$\displaystyle{\lim \limits_{x \to a}\frac{f(x)-f(a)}{x-a}=f'(x)}$ \\ %\displaystyle で文字を見やすくする

$\displaystyle{\int \sin x \,dx=-\cos x + C}$ \\ %"\,"でスペースが空く。積分は\int。

$\int_a^b$ \\

$\int \limits_a^b$

$\displaystyle{\int_a^b}$ \\

$\displaystyle{\int_{2a}^b}$ \\
$\displaystyle{\int_a^b x^2\,dx = \left[\frac{x^3}{3}\right]_a^b = \frac{b^3}{3} - \frac{a^3}{3}}$ \\ 

$\displaystyle{\sum_{n=1}^{\infty}ar^n=a+ar+ar^2+ \cdots +ar^n+ \cdots}$ \\ %…は\cdots。・は\cdot

$\displaystyle{ \int_a^b f(x) \,dx = \lim \limits_{x \to \infty} \sum_{k=1}^{n} f(x_k) \cdot \Delta x }$ \\ %\deltaと\Deltaは別

$\vec{v}=v_1 \vec{i}+v_2 \vec{j}= \langle v_1, v_2 \rangle$ %\vec{•}で矢印付のベクトル

\end{document}