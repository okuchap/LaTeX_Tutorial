\documentclass[11pt,a4paper]{amsart}

\theoremstyle{definition}
\newtheorem{thm}{Theorem}
\newtheorem{thm*}{Theorem}
\newtheorem{lem}{Lemma}
\newtheorem{prop}{Proposition}
\newtheorem{ex}{Example}
%
\theoremstyle{definition}
\newtheorem{defn}{Definition}[section]
%
\theoremstyle{remark}
\newtheorem{rem}{Remark}
\newtheorem{prf}{Proof}
\renewcommand{\theprf}{}

\usepackage{amsthm}
\usepackage[utf8]{inputenc}
\usepackage{amsmath}
\usepackage{amsfonts}
%\usepackage{amssymb}
\usepackage{graphicx}
\usepackage[left=2cm,right=2cm,top=2cm,bottom=2cm]{geometry} %This changes the margins.
\begin{document}

\title{Oscillation of solutions}
\author{Kyohei OKUMURA}
\date{\today}

\maketitle
\section{Basic Results}
%\subsection{}

Through this article, $y$ is a function of $x$. We are going to tackle with the ODE below:
\begin{equation}
\displaystyle{ [p(x)y']' + q(x)y = 0} 
\end{equation}
,where $p,q$ are bounded and continuous on an interval $I$ and $\forall x \in I : \ p(x)>0.$

\begin{lem}
All second order variable coefficient linear ODE i.e. $y'' + g(x)y' + h(x)y = 0 $ can be transformed into the above form, where $g$ and $h$ are bounded continuous functions defined on an interval $I$.
\end{lem}
\begin{prf}
Take an arbitrary constant $c \in I$. Define $p$ and $q$ as follows:
$$p(x):=\exp[\int_c^x g(t)dt]$$
$$q(x):=h(x)p(x)$$
\qed
\end{prf}

\begin{thm}[Sturm Separation Therem]
Suppose $u$ and $v$ are two linearly independent solutions of (1). Then, zeros of $u$ and $v$ occur in turn. More precisely, there must be a zero of $u$ between two zeros of $v$, and vice versa. 
\end{thm}
This theorem shows that all nontrivial solutions of (1) oscillate at the same speed.

Next, Think of two different ODEs like these:
\begin{equation}
\displaystyle{ [p_1(x)y']' + q_1(x)y = 0}
\end{equation}
\begin{equation}
\displaystyle{ [p_2(x)y']' + q_2(x)y = 0}
\end{equation}
We would like to think about the following question: Which of the solutions of these two ODEs oscillates faster?

Suppose $y$ is a solution of (1). By using polar coordinates, we can obtain these representations:
\begin{equation}
\displaystyle{ y = \rho \sin \theta}
\end{equation}
\begin{equation}
\displaystyle{ p(x)y' = \rho \cos \theta}
\end{equation}
, where $\rho$ and $\theta$ are functions of $x$.(By the way, I am not sure why we can always get this representation...) 

\begin{prop}
The following ODE about $\theta$ holds:
\begin{equation}
\displaystyle{ \theta' = \frac{1}{p(x)}\cos^2\theta + q(x)\sin^2\theta }
\end{equation}
\end{prop}
\begin{prf}
By differentiating (4) by $x$, we can obtain
$$y'=\rho' \sin \theta + \rho \cos \theta \cdot \theta'$$
Then, By (5), The following equation holds.
\begin{equation}
\rho' \sin \theta + \rho \cos \theta \cdot \theta' = \frac{1}{p(x)} \rho \cos \theta
\end{equation}

In the same way, by differentiating (5) by $x$,

$$[p(x)y']' = \rho' \cos \theta - \rho \sin \theta \cdot \theta'$$

By (1) and (4),

\begin{equation}
\rho' \cos \theta - \rho \sin \theta \cdot \theta' = -q(x)\rho \sin \theta
\end{equation}

Finally, by calculating (7) $\times \cos \theta$ - (8) $\times \sin \theta$, we can obtain the result.
\qed
\end{prf}

\section{Theorems}

By using (6), we will compare the speed of oscillation of the two equations, (2) and (3).
Apply Prufer transformation to the solutions of (2) and (3). The equation about their arguments are as follows:

\begin{equation}
\displaystyle{ \theta_1' = \frac{1}{p_1(x)}\cos^2\theta_1 + q(x)\sin^2\theta_1 }
\end{equation}

\begin{equation}
\displaystyle{ \theta_2' = \frac{1}{p_2(x)}\cos^2\theta_2 + q(x)\sin^2\theta_2 }
\end{equation}

The following theorem, known as Sturm Comparison Theorem, holds.

\begin{thm}[Sturm Comparison Theorem]

Suppose $\theta_1$ and $\theta_2$ are the solutions of (9) and (10).

Assume the following properties hold:

\begin{enumerate}
\item On the interval $[a,b]$, $p_1(x),q_1(x),p_2(x)$ and $q_2(x)$ are continuous.
\item $p_1(x) \geq p_2(x) > 0$, and $q_1(x) \leq q_2(x)$
\item $\theta_1(a) \leq \theta_2(a)$
\end{enumerate}

Then,

$$\forall x \in [a,b]; \quad \theta_1(x) \leq \theta_2(x)$$

Moreover, in addition to the three assumptions above, if either $p_1(x) > p_2(x)$ or $q_1(x) < q_2(x)$ holds for all $x$ in $(a,b)$, then

$$\forall x \in (a,b]; \quad \theta_1(x) < \theta_2(x)$$

\end{thm}

This theorem shows that the solution of (1) oscillates faster if $p$ is smaller or $q$ is bigger.

\section{Examples I couldn't understand}
\begin{ex}
Assume the followings:

\begin{itemize}
	\item $p$ and $q$ are defined on $x>0$ and continuous.
	\item $\exists L,M > 0 \quad \forall x; \quad 0<p(x)<L, \quad q(x)>M $
\end{itemize}

Show that in this case all nontrivial solutions of (1) have infinitely many zeros.

\begin{prf}
Think of the following ODE:
$$Lz'' + Mz = 0$$
Its solution is:
$$z = C_1 \sin (\omega + C_2) \quad (\omega := \sqrt{\frac{M}{L}})$$
Compare the equation (1) and this ODE. By using Sturm Comparison Theorem, $y$ oscillates faster than $z$(???). Because $z$ has infinitely many zeros, $y$ also does.
\qed
\end{prf}
\end{ex}

\begin{ex}
Give the proof of Sturm Separation Theorem by using Sturm Comparison Theorem.
\begin{prf}
Aplly Prufer transformation to two solutions. Then, we can satisfy the condition $\theta_1 \leq \theta_2 \leq \theta_1 + \pi$.(???)
\end{prf}
\end{ex}

\begin{ex}
Show that all nontrivial solutions of the following ODE has at most two zeros.
$$y'' + \frac{1}{4(1+x^2)}y = 0$$

\begin{prf}
Note that $y:=\sqrt{|x|}$ is the solution of $\displaystyle{y'' + \frac{1}{4x^2} = 0}$. Use Sturm Comparison Theorem. We can see that the given ODE can have at most one zero in $x>0$ and at most one zero in $x<0$.(???)
\end{prf}
\end{ex}

\begin{ex}
Assume the followings:
\begin{itemize}
	\item $p$ and $q$ are defined on $x>0$ and continuous.
	\item $\displaystyle{\lim_{x \to \infty}=\alpha>0, \quad \lim_{x \to \infty}=\beta>0}$
	\item $x_1<x_2<x_3<\cdots$ are zeros of a nontrivial solution of (1).
\end{itemize}

Show that the following equation holds:

$$\displaystyle{\lim_{x \to \infty}(x_{n+1}-x_n)=2\pi \sqrt{\frac{\alpha}{\beta}}}$$

\begin{prf}
Compare the oscillations of (1) and $(\alpha \pm \epsilon)z'' + (\beta \mp \epsilon)z = 0$(???)
\end{prf}

\end{ex}

\end{document}  