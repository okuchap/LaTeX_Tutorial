\documentclass[12pt]{jsarticle}
\usepackage[dvipdfm]{graphicx}
\usepackage{pxfonts}
\pagestyle{empty}

%% 高さの設定
\setlength{\textheight}{\paperheight}   % ひとまず紙面を本文領域に
\setlength{\topmargin}{-5.4truemm}      % 上の余白を20mm(=1inch-5.4mm)に
\addtolength{\topmargin}{-\headheight}  % 
\addtolength{\topmargin}{-\headsep}     % ヘッダの分だけ本文領域を移動させる
\addtolength{\textheight}{-40truemm}    % 下の余白も20mmに
%% 幅の設定
\setlength{\textwidth}{\paperwidth}     % ひとまず紙面を本文領域に
\setlength{\oddsidemargin}{-5.4truemm}  % 左の余白を20mm(=1inch-5.4mm)に
\setlength{\evensidemargin}{-5.4truemm} % 
\addtolength{\textwidth}{-40truemm}     % 右の余白も20mmに
\begin{document}


\title {TeXサンプル}
\author{Chaos Lab.}
\date{2012.11.25}
\maketitle

\section{コーシーの積分表示}
\subsection{コーシーの積分表示}
$f(z)$は領域$D$で正則であるとする。$D$内に単一閉曲線$C$があるとき、点$a$が$C$の内部にあるならば次式が成り立つ。
\begin{equation}
f(a)=\frac{1}{2\pi i}\int _C \frac{f(z)}{z-a}dz
\end{equation}
\subsection{練習問題}
次の積分を求めよ。ただし$C$は$|z|=2$とする。
\begin{equation}
\int _C\frac{z^3}{z+i}dz
\end{equation}
\subsubsection*{$<$解答$>$}
点$z=-i$は円$C$の内部にあり\footnote{$\frac{z^3}{z+i}=\frac{z^3}{z-(-i)} 虚数軸上の点(0,-i)は円C:|z|=2の内部にある。$}、
$f(z)=z^3$は$C$の内部で正則であるから、コーシの積分表示から
\begin{eqnarray*}
\frac{1}{2\pi i}\int _C\frac{z^3}{z+i}dz=f(-i)=(-i)^3&=&i\\
よって   \int _C\frac{z^3}{z+i}dz=2\pi i^2&=&\underline{-2\pi}
\end{eqnarray*}

\end{document}