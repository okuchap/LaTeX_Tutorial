\documentclass[11pt,a4paper]{article}
\usepackage[utf8]{inputenc}
\usepackage{amsmath}
\usepackage{amsfonts}
%\usepackage{amssymb}
\usepackage{graphicx}
\usepackage[left=2cm,right=2cm,top=2cm,bottom=2cm]{geometry} %This changes the margins.
\author{Kyohei Okumura}
\begin{document}

\title{数理統計テキスト正誤表}
\author{Kyohei Okumura}
\date{\today}
\maketitle

\section*{第2章}

\begin{itemize}
\item p.15 例2.12

「k回目に表が出たら」→「k回目に初めて表が出たら」
\end{itemize}

\section*{第3章}
\begin{itemize}

\item p.21 3.1.1 離散一様分布 Nで割るのを忘れている。
$$E[X] = \displaystyle{\sum_{x=1}^N {\frac{x}{N}} = \frac{N+1}{2}, E[X^2] = \sum_{x=1}^N \frac{x^2}{N} = \frac{(N+1)(2N+1)}{6}}$$
が正しいと思います。

\item p.25 命題3.7 証明

下から4行目(最後のkは不要)
$$\displaystyle{\frac{(r-1)!}{(1-q)^r} = \sum_{k=1}^\infty (k+r-1) \cdots (k+1)q^k}$$
下から2行目(同上)
$$1 = \displaystyle{\sum_{k=1}^\infty \frac{(k+r-1) \cdots (k+1)}{(r-1)!}p^r q^k}$$
がそれぞれ正しいと思います。

\end{itemize}

\section*{第4章}
\begin{itemize}

\item 相関係数の定義の後 $Corr(X,Y)$が抜けてる?

「実数$a,b,c,d$に対して~、依存してしまう。これに対して、相関係数は$Corr(aX+b, cY+d) = (ac/|ac|) Corr(X,Y)$となるので、~」

が正しいのではないでしょうか。


\end{itemize}

\section*{第5章}
\begin{itemize}

\item p.60 定理5.20の証明直後

「定理5.20において$\sigma^2$が~」→「定理5.19において$\sigma^2$が~」

\item p.63 (5.18)より2行下

「極限分布は1点になってします。」→「1点になってしまう。」

\end{itemize}

\section*{第6章}
\begin{itemize}

\item p.70 定義6.1の直後 $T(X)=t$とすべき箇所が$T(X)=x$になっている。

「離散確率変数の場合~、同時確率は、
$P_\theta (X=x) = P_\theta (X=x, T(X)=t) = P_\theta (X=x | T(X)=t)$
」
が正しいと思います。

\item p.78 medianの期待値 二つ目の変形のマイナスは不要
$$E[med(X_1, \ldots, X_n) - \mu] = E[med(Z_1, \ldots, Z_n)] = E[med(-Z_1, \ldots, -Z_n)] = -E[med(Z_1, \ldots, Z_n)]$$
が正しいと思います。

\end{itemize}

\section*{第7章}
\begin{itemize}

\item p.86 一番下の行の最後 $z_\alpha$ではなく$\alpha$
$$P_{H_0}(|\bar{X} - \bar{Y}| / \sigma_0 > C) = P_{H_0}(|Z| > C\sqrt{mn} / \sqrt{m+n}) = \alpha$$
が正しいと思います。

\item p.90 下から9行目

「なる量をが検定統計量として~」→「なる量を検定統計量として~」


\end{itemize}


\end{document}