\documentclass[11pt]{article}

\begin{document}

$$(x+1)$$
$$[2+(x+1)]$$
$$\{a,b,c\}$$ % {や$を表示させたいときは、\を左につける。
$$\$12.55$$

$$3\left(\frac{2}{5}\right)$$ %(や[のサイズ調整は\left,\right
$$3\left[\frac{2}{5}\right]$$
$$3\left\{\frac{2}{5}\right\}$$

$$|x|$$
$$\left|\frac{x}{x+1}\right|$$

$$\left\{x^2\right.$$
$$\left. \frac{dy}{dx} \right|_{x=1}$$ %(や|などを片方だけ表示したいときは、.で置換

\begin{tabular}{|c|c|c|c|c|c|} %cの数が行数。表に縦線を入れるときはcの間に|を挟む。
\hline
$x$ & 1 & 2 & 3 & 4 & 5 \\ \hline %\\で改行。\hlineで横線。
$f(x)$ & 10 & 11 & 12 & 13 & 14 \\ \hline
\end{tabular}

\begin{eqnarray*} %式を並べる
5x^2-9&=&2x+3\\ %"="を"& ampersand"で挟むと、式が揃う。
4x^2&=&12\\
x^3&=&3\\
x&\approx&\pm1.732
\end{eqnarray*} %"*"を付けると、式番号が非表示に

\end{document}